\documentclass{jsarticle}
\usepackage[dvipdfmx]{graphicx}
\begin{document}

\title{身体動作と筋電量の関係性}
\author{平松亨隆}
\maketitle


\section{目的}
人は、意識的に筋肉の伸縮を考えて体を動かすわけではない。例えば、上腕二頭筋を使うことを意識して腕を曲げる人はまずいないだろう。だから、私達はどのような筋活動で運動をしているのかということに興味を持った。そこで、今回は腕の屈伸運動に着目し、上腕二頭筋と上腕三頭筋がどのような筋活動をしてその運動をしているのか、を明らかにすることを本レポートの目的とした。しかしこのような反復運動の場合、運動の速度を上げるほどブレて制御がしにくくなるという問題があるため、速度に違いがある場合には筋活動も違う可能性がある。そのため、腕の屈伸運動は二つの速度で行うことにした。

\section{実験方法}
\subsection{運動計測}
下記の図\ref{fig:short}はもっとも腕を曲げた状態、図\ref{fig:long}は最も腕を伸ばした状態の図である。被験者にはこの2つの状態を繰り返す腕の曲げ伸ばし運動を、利き腕を使って、速度に関して何も指示を出していない場合と速く動かしてもらった場合の2パターンで計測をした。被験者には反射マーカーを肩、肘、手首につけてモーションキャプチャー(ライブラリ社:MoveTR)を使って1000 fpsで撮影した。また筋電センサ(ロジカルプロダクト社)を上腕二頭筋と上腕三頭筋に貼付し、筋電位をサンプリング周波数200 Hzで計測した。筋電センサとカメラの計測時間は同期をとり、20秒間計測していた。
\begin{figure}[h]
  \begin{minipage}{0.5\hsize}
    \begin{center}
      \includegraphics[width=7cm]{images/short.jpg}
    \end{center}
    \caption{腕を縮めた状態}
    \label{fig:short}
  \end{minipage}
  \begin{minipage}{0.5\hsize}
    \begin{center}
      \includegraphics[width=7cm]{images/long.jpg}
    \end{center}
    \caption{腕を伸ばした状態}
    \label{fig:long}
  \end{minipage}
\end{figure}

\subsection{解析}
\subsubsection{筋電位}
筋電位のデータにはいくつかの処理を順番に施して、筋肉の活動度を求めた。
  まずデータから筋電位の特徴を抽出するために、通過帯域を1〜40 Hzとしたバンドパスフィルタ(3次バターワースフィルタ)をかけた。しかし、バンドパスフィルタをかけることによってデータに位相ズレが生じるため、時間軸を反転してもう一度同じフィルタをかけることによってこれを補正した。そしてフィルタ後のデータを整流し、時刻$t$の筋電位$|E(t)|$に対して時間$\Delta T$の間で平均を求め、筋肉の活動度とした。式は以下を用いた。
  \begin{equation}
    a(t) = \frac{1}{\Delta T} \int_{t-\frac{\Delta T}{2}}^{t+\frac{\Delta T}{2}} |E(t)| dt
  \end{equation}

\section{結果}
\begin{figure}[b]
  \begin{center}
    \includegraphics[width=15cm]{images/s2proto.png}
  \end{center}
  \caption{速い時の運動と筋肉の活動度の関係}
  \label{fig:fast}
\end{figure}
図\ref{fig:fast}は、腕を速く動かしてもらった場合である。縦軸は身体に対する手首の前後方向の変位と筋肉の活動度、横軸は時間である。手首の前後方向の変位は値が大きくなるほど体の前に伸ばしていることを示す。図中の紫、緑、青の線はそれぞれ身体に対する手首の前後方向の変位、上腕二頭筋肉の活動度、上腕三頭筋の活動度である。手首の前後方向の変位が、横方向や縦方向に比べて変位が大きく、動きが把握しやすかったからである。このグラフでは、運動開始後時刻8〜12秒の間を表示している。

図\ref{fig:fast}から、例えば時刻8.8秒の時、腕がもっとも屈曲した状態では上腕三頭筋の活動度は大きい。これは腕が停止した状態から伸展方向に加速度を与える必要があるからである。逆に、屈曲した状態から腕を伸展させるために屈筋である上腕二頭筋[1]は必要ない。だからこの時の上腕二頭筋活動度は小さい。しかし、この状態から腕が伸展を始めると同時に二頭筋の活動度が大きくなっていっている。これは、上腕三頭筋による腕の伸展運動を減速や停止など、制御するための筋活動だろう。そして腕を伸展させきったところで上腕三頭筋の活動度は小さくなる。これは、伸展した状態から腕を屈曲させるために伸筋である上腕二頭筋[1]は必要ないからである。この後、腕を屈曲させ始めるとしばらくして上腕二頭筋の活動度が下がり始める。これは、上腕二頭筋が屈曲方向に大きな加速度を与える必要があるのは屈曲運動の最初期だけであり、あとは慣性によって大きな加速度を与える必要がなくなるからである。このとき上腕三頭筋は活動度を大きくしていっている。これは、屈曲運動をする際、もっとも筋活動が必要なのが直前の伸展運動の停止であり、それ以降は屈曲運動に速度を与えるだけのため、あまり筋活動をする必要がないからである。このとき上腕三頭筋は値を大きくしていっている。これは、上腕二頭筋による腕の屈曲運動を減速や停止など、制御するための筋活動である。
また、時刻8.8の時に上腕二頭筋の活動度が極小、上腕三頭筋の活動度は極大。時刻9.1の時は上腕二頭筋の活動度が極大、上腕三頭筋の活動度が極小。というようにこのグラフでは上腕二頭筋と三頭筋の活動度は逆位相の形をとっている。

\begin{figure}[b]
  \begin{center}
    \includegraphics[width=15cm]{images/s1proto.png}
  \end{center}
  \caption{速度に関する指示のない運動と筋肉の活動度の関係}
  \label{fig:slow}
\end{figure}
図\ref{fig:slow}は、速度に関して指示を出さず腕を動かしてもらった場合である。各軸や凡例は図\ref{fig:fast}と同様である。このグラフでは、運動開始後時刻5〜15秒の間を表示している。例えば時刻10秒の時は、腕が屈曲した状態である。このとき腕が伸展を始め、少ししてからまず上腕三頭筋の活動度が大きくなっていっている。上腕三頭筋は伸筋であるため、これは腕を伸展させる筋活動である。次に、この少し後に上腕二頭筋が活動度が大きくなっていっている。これは腕がもっとも伸展した状態の前から大きくなっている。上腕二頭筋は屈筋であるため[1]、伸展の逆方向の加速度を与えて速度を小さくしているのである。伸展し終わり、屈曲が始まってしばらく経つと上腕二頭筋が活動度を下げ始める。これは上腕二頭筋が屈曲運動をする際、もっとも筋活動が必要なのが直前の伸展運動の停止であり、それ以降は屈曲運動に速度を与えるだけのため、あまり筋活動をする必要がないからである。このとき上腕三頭筋は腕を屈曲しきる少し前まで筋活動をしているが、屈曲しきる頃にはその活動度を下げている。これは腕を速く動かした時と異なる。時刻10の時、上腕二頭筋と三頭筋の活動度は共に極小である。そして時刻10.6の時は上腕二頭筋の活動度は極大。上腕三頭筋の活動度は少しでこぼこしているがほぼ極大である。このことからこのグラフでは上腕二頭筋と三頭筋の活動度はほぼ同位相である。

図\ref{fig:slow}と図\ref{fig:fast}を比較すると、速度に関する指示を出さない場合(図\ref{fig:slow})では同じ運動にも関わらず、上腕二頭筋と上腕三頭筋の活動度はほぼ同じ位相、速く動かしてもらった場合(図\ref{fig:fast})では逆位相と、全く違う結果になっている。これは一体どういうことなのだろうか。

\section{考察}
自然に腕を動かしてもらったときと速く腕を動かしてもらった時、上腕三頭筋と上腕二頭筋の位相が異なっていたのは単純に速度の違いによって発生したと推測される。例えば、腕を速く動かすときの屈曲運動では当然腕が持つ速度は大きいため、屈曲しきる時の上腕三頭筋の筋活動量はたくさん必要である。しかし、腕を自然な速度で動かすときは腕が持つ速度は大きくないため、屈曲しきるときの上腕三頭筋の筋活動量はあまり必要ではない。そのため、腕が屈曲しきった時には図\ref{fig:fast}では上腕三頭筋の筋活動が極大であり、図\ref{fig:slow}では上腕三頭筋の筋活動が極小である。
また、腕をもっとも伸展させた状態ではどのようになるかを考える。先ほどの屈曲運動の時と同様に考えれば伸展運動の際に伸展方向とは逆方向の加速度を与える上腕二頭筋の筋活動は小さいはずである。しかし、例えば図\ref{fig:slow}の時刻10.5のとき、腕がもっとも進展した状態であるのに上腕二頭筋の活動度は大きい。一体どうなっているのだろうか。このとき、腕の状態がどこまで伸展しているのかを表したのが図\ref{fig:arm}である。この図\ref{fig:arm}より、腕を自然な速度で動かすとき、腕をもっとも伸展させた状態でも完全に腕を伸展させきっているわけではないのである。腕の屈伸運動を考えるとき、腕を屈曲させるときは完全に屈曲しきる前に上腕と前腕がぶつかることによって上腕三頭金の筋活動によらない速度の低下が起こっている。しかし、腕を自然な速度で動かすとき、腕の伸展運動は腕を完全に伸展させる前の状態で停止させている。そのため、腕の伸展運動の減速、停止に必要な筋活動は屈伸運動のそれより大きいのだろう。

\begin{figure}[b]
  \begin{center}
    \includegraphics[width=15cm]{images/armmodel.png}
  \end{center}
  \caption{速度に関する指示のない運動と筋肉の活動度の関係}
  \label{fig:arm}
\end{figure}


\section*{参考文献}
[1] 河合 良訓 原島 広至, Chapter 3 上肢, In: 肉単 〜語源から覚える解剖学英単語集〜,株式会社エヌ・ティー・エス,P54〜P55,2004年
\end{document}