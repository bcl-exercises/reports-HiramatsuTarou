\documentclass{jsarticle}
\usepackage[dvipdfmx]{graphicx}
\begin{document}

\title{身体動作と筋電量の関係性}
\author{平松亨隆}
\maketitle


\section{目的}
人は、意識的に筋肉の伸縮を考えて体を動かすわけではない。例えば、上腕二頭筋を使うことを意識して腕を曲げる人はまずいないだろう。だから、私達はどのような筋活動で運動をしているのかということに興味を持った。そこで、今回は腕の屈伸運動に着目し、上腕二頭筋と上腕三頭筋がどのような筋活動をしてその運動をしているのか、を明らかにすることを本レポートの目的とした。しかしこのような反復運動の場合、運動の速度を上げるほどブレて制御がしにくくなるという問題があるため、速度に違いがある場合には筋活動も違う可能性がある。そのため、腕の屈伸運動は二つの速度で行うことにした。

\section{実験方法}
\subsection{運動計測}
下記の図\ref{fig:short}はもっとも腕を曲げた状態、図\ref{fig:long}は最も腕を伸ばした状態の図である。被験者にはこの2つの状態を繰り返す腕の曲げ伸ばし運動を、利き腕を使って、速度に関して何も指示を出していない場合と速く動かしてもらった場合の2パターンで計測をした。被験者には反射マーカーを肩、肘、手首につけてモーションキャプチャー(ライブラリ社:MoveTR)を使って1000 fpsで撮影した。また筋電センサ(ロジカルプロダクト社)を上腕二頭筋と上腕三頭筋に貼付し、筋電位をサンプリング周波数200 Hzで計測した。筋電センサとカメラの計測時間は同期をとり、20秒間計測していた。
\begin{figure}[h]
  \begin{minipage}{0.5\hsize}
    \begin{center}
      \includegraphics[width=7cm]{images/short.jpg}
    \end{center}
    \caption{腕を縮めた状態}
    \label{fig:short}
  \end{minipage}
  \begin{minipage}{0.5\hsize}
    \begin{center}
      \includegraphics[width=7cm]{images/long.jpg}
    \end{center}
    \caption{腕を伸ばした状態}
    \label{fig:long}
  \end{minipage}
\end{figure}

\subsection{解析}
\subsubsection{筋電位}
筋電位のデータにはいくつかの処理を順番に施して、筋肉の活動度を求めた。
  まずデータから筋電位の特徴を抽出するために、通過帯域を1〜40 Hzとしたバンドパスフィルタ(3次バターワースフィルタ)をかけた。しかし、バンドパスフィルタをかけることによってデータに位相ズレが生じるため、時間軸を反転してもう一度同じフィルタをかけることによってこれを補正した。そしてフィルタ後のデータを整流し、時刻$t$の筋電位$|E(t)|$に対して時間$\Delta T$の間で平均を求め、筋肉の活動度とした。式は以下を用いた。
  \begin{equation}
    a(t) = \frac{1}{\Delta T} \int_{t-\frac{\Delta T}{2}}^{t+\frac{\Delta T}{2}} |E(t)| dt
  \end{equation}

\section{結果}
\begin{figure}[b]
  \begin{center}
    \includegraphics[width=15cm]{images/s2proto.png}
  \end{center}
  \caption{速い時の運動と筋肉の活動度の関係}
  \label{fig:fast}
\end{figure}
図\ref{fig:fast}は、腕を速く動かしてもらった場合である。縦軸は身体に対する手首の前後方向の変位と筋肉の活動度、横軸は時間である。手首の前後方向の変位は値が大きくなるほど体の前に伸ばしていることを示す。図中の紫、緑、青の線はそれぞれ身体に対する手首の前後方向の変位、上腕二頭筋肉の活動度、上腕三頭筋の活動度である。手首の前後方向の変位が、横方向や縦方向に比べて変位が大きく、動きが把握しやすかったからである。このグラフでは、運動開始後時刻8〜12秒の間を表示している。

図\ref{fig:fast}から、例えば時刻8.8秒の時、腕がもっとも屈曲した状態では上腕三頭筋の活動度は大きい。これは腕が停止した状態から伸展方向に加速度を与える必要があるからである。逆に、屈曲しきったこの時期には屈筋である上腕二頭筋[1]の活動は必要ない。だからこの時の上腕二頭筋活動度は小さい。しかし、この状態から腕が伸展を始めると同時に二頭筋の活動度が大きくなっていっている。これは、上腕三頭筋による腕の伸展運動を減速や停止するための筋活動だろう。そして腕を伸展しきったところで上腕三頭筋の活動度は小さくなる。これは、この伸展しきった時期に伸筋である上腕二頭筋[1]は必要ないからである。この後、腕を屈曲させ始めるとしばらくして上腕二頭筋の活動度が下がり始める。これは、上腕二頭筋が屈曲方向に大きな加速度を与える必要があるのは屈曲運動の最初期だけであり、あとは慣性によって大きな加速度を与える必要がなくなるからである。このとき上腕三頭筋は活動度を大きくしていっている。これは、屈曲運動をする際、もっとも筋活動が必要なのが直前の伸展運動の停止であり、それ以降は屈曲運動に速度を与えるだけのため、あまり筋活動をする必要がないからである。このとき上腕三頭筋は値を大きくしていっている。これは、上腕二頭筋による腕の屈曲運動を減速や停止など、制御するための筋活動である。
また、時刻8.8秒の時に上腕二頭筋の活動度が極小、上腕三頭筋の活動度は極大。時刻9.1秒の時は上腕二頭筋の活動度が極大、上腕三頭筋の活動度が極小、というように上腕二頭筋と三頭筋の活動度は逆位相の形をとっている。

\begin{figure}[b]
  \begin{center}
    \includegraphics[width=15cm]{images/s1proto.png}
  \end{center}
  \caption{速度に関する指示のない運動と筋肉の活動度の関係}
  \label{fig:slow}
\end{figure}
図\ref{fig:slow}は、速度に関して指示を出さず腕を動かしてもらった場合である。各軸や凡例は図\ref{fig:fast}と同様である。このグラフでは、運動開始後時刻5〜15秒の間を表示している。例えば時刻12.3秒の時は、腕が屈曲した状態である。このとき上腕二頭筋と上腕三頭筋の活動度は極小ではないが小さい。その後、腕を伸展させ始めると上腕三頭筋、上腕二頭筋の順に極小と活動度の増加を迎える。そして伸展しきったて屈曲を始めた後、今度は上腕二頭筋、上腕三頭筋の順に活動度の低下が始まる。このように上腕二頭筋も上腕三頭筋も極大極小の位置が腕を速く動かす場合(図\ref{fig:fast})とは異なっている。

図\ref{fig:slow}と図\ref{fig:fast}を比較すると、同じ運動にも関わらず速く動かしてもらった場合(図\ref{fig:fast})では逆位相、速度に関する指示を出さない場合(図\ref{fig:slow})では逆位相から大きくずれた形、と全く違う結果になっている。これは一体どうなっているのだろうか。

\section{考察}
自然に腕を動かしてもらったときと速く腕を動かしてもらった時、上腕三頭筋と上腕二頭筋の位相が異なっていたのは速度の違いや腕の構造によって発生したと推測される。腕を速く動かすときはとにかく腕を速く動かせばいいため、速度を考えて腕を動かす必要はない。しかし、自然な速度で腕を動かすとき、遅くなりすぎたり速くなりすぎたりしないように腕の速度を常に気にする必要がある。だから、自然な速度で腕を動かすときは上腕二頭筋と上腕三頭筋の力を釣り合わせるようにして等速に近い運動をしている可能性がある。例えば図\ref{fig:slow}の時刻12.3秒の時は、腕が屈曲した状態である。このとき上腕三頭筋の活動度は小さい。これは、腕を自然に動かす場合は、あまり大きな速度を腕が持っていないことと、また腕を屈曲させた際に前腕と上腕が衝突して速度が低下すること、という2つの理由で屈曲しきる直前の減速、停止させるための筋活動があまり必要ないからである。このとき上腕二頭筋の筋活動も小さいが、これは前項で説明と同じで上腕二頭筋が屈筋であるからである。その後、屈曲状態から伸展させ始めると、まず上腕三頭筋の活動度が大きくなり始める。これは上腕三頭筋が伸筋であるからである。次に上腕三頭筋の活動度が大きくなり始める。これは、上腕三頭筋による腕の伸展運動を制御するための筋活動である。その後伸展しきった際、上腕二頭筋の活動度が大きい状態である。これは、腕を屈曲させ切る際のような前腕と上腕の衝突はないため、腕を減速、停止させるための上腕二頭筋の筋活動が必要だったからである。そしてこのとき力を吊り合わせて速度を制御するために上腕三頭筋の活動度も大きい。
以上によって腕を自然に動かす場合と速く動かす場合の違いが生まれているのだろう。

\section*{参考文献}
[1] 河合 良訓 原島 広至, Chapter 3 上肢, In: 肉単 〜語源から覚える解剖学英単語集〜,株式会社エヌ・ティー・エス,P54〜P55,2004年
\end{document}